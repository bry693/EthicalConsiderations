\documentclass[10pt,twocolumn]{article}
\usepackage{oxycomps}
\usepackage{hanging}
\usepackage [english]{babel}
\usepackage [autostyle, english = american]{csquotes}
\MakeOuterQuote{"}
\pdfinfo{
    /Title (Ethical Concerns of a Digital Scene Design Tool)
    /Author (Bryanna Hernandez)
}
\title{Ethical Concerns of a Digital Scene Design Tool}
\author{Bryanna Hernandez}
\affiliation{Occidental College}
\email{bhernandez2@oxy.edu}

\begin{document}

\maketitle

\section{Introduction}
As a Computer Science major with a concentration in worldbuilding, I wanted to design a comprehensive senior project that allows theatre set designers and similar creatives to build three-dimensional scenes. The purpose of this tool was to be a useful visual aid for students and amateur designers to refine their artistic vision. At first glance, this idea seems essentially harmless and may be in some circumstances useful as a tool to distribute creative power to artists with less experience or resources.  

However, on further examination, there are subtle issues with designing software of this nature that I would be unable to address over the few short months I am allotted to build it. Therefore, since I cannot effectively address any ethical issues in the construction of this project, I would like to contend with these issues in writing. Through this format, I hope to comprehensively acknowledge the harmful complications and pitfalls that my project could result in and come to a conclusion about whether my set designing software's pros outweigh the cons to decide if it is still a worthwhile project. 
\section{Potential User Abuse}
One of the most obvious issues with my project is the potential for user abuse. When designing a tool that is meant to hand creative control over to the user, this is an inevitable problem. As an example, for a program like Photoshop which belongs to the Adobe Suite, there is a condition within the General Terms of Use that prohibits its consumers from generating content that is "unlawful, harmful, threatening, obscene, violent, abusive," or in any way objectionable (“Legal”). The company Electronic Arts (EA) has a similar user agreement which prohibits the distribution of content that is "privacy-invasive, vulgar, offensive, indecent or unlawful" (“ELECTRONIC ARTS USER AGREEMENT”).

While it would be simple enough for me to implement a user agreement that requires an electronic signature, the difficulty lies in the regulation of behavior and the ability to address and punish users who have disrespected the agreement. Within my project, there would be a function that allows users to upload props created in other software such as Blender. However, I would be unable to implement a function to screen any user-created content. This means there is a possibility that users could upload harmful props such as a Nazi swastika or vulgar objects like phallic depictions. Additionally, at my level of expertise, it would be nigh impossible for me to take action against consumers who have misused the tool even if I could identify them if, for example, a user posted a screenshot of their set. 

On the other hand, more powerful companies like Adobe have the ability to screen for abusive content and "reserve the right to to remove Content or restrict access to Content" if the content is found to be in violation (“Legal”). EA has a similar policy of monitoring accounts for rule violations and "revoking access to certain or all EA Services," (“ELECTRONIC ARTS USER AGREEMENT”). These are responsible actions that ensure the safety of its users and the reputation of the company. If I am unable to enact similar punishments in response to user abuses, then there is a possibility that my project could be used to harass and/or harm others. It would be irresponsible of me to ignore this outcome.

The simplest way to prevent user abuses would therefore be to disallow users to upload content. However, limiting this function results in another issue regarding the restriction of artistic expression. The purpose of my software is to create an environment that is intuitive and efficient for designers to build and view their artistic visions, and I would inevitably fail to provide pre-built props that satisfy every set. 
\section{Technological Solutionism}
Technological Solutionism is essentially the arrogance of believing that every problem in our society has a solution if only one can create the right algorithm. As the creator of this term, Evgeny Morozov, acknowledges in an interview with Natasha Dow Schüll that the issue doesn't come from utilizing technology as a new part of our "problem-solving apparatus". Instead, the issues lie in the pursuit of reducing complex problems down to the numbers without "any holistic understanding"(Dow Schull). This results in a problematic understanding of a situation with a single, uniform solution for all people without considering the inherent diversity of their needs.

In the practice of using computer science to provide a creative tool, it is therefore imperative to ensure that the software doesn't force its users to conform to a singular practice or attitude regarding creative expression. Edwin Creely and Danah Henriksen's paper \textit(Creativity and Digital Technologies) examines the role and advantages of digital tools, positing that they provide safe spaces for "managed risk-taking and enacting constructive failure"(Creely and Henriksen). For my project, this is most relevant when considering the lack of physical costs regarding creative risks, as the software allows for an in-between stage between inception and material realization. 

However, the software is only suitable for a limited use of experimentation. In the supposition that creativity requires a sufficient balance of "embodied play, touch, and interpersonal interactions" my project would be unable to satisfy those requirements(Creely and Henriksen). Starting with the idea of embodied play which requires a tangible usage of the corporeal body, interaction through a digital screen would at best be able to simulate physical connection through haptic controllers such as the Playstation Five controller, which vibrates and responds to player breath in certain situations. This type of haptic interaction is completely beyond the scope of my project. I cannot imagine a single implementation at the end of the semester that utilizes any other device than a computer with a mouse.  

As an extension of this subject, the visual nature of software would also face issues of accessibility for blind or low-vision individuals. For non-tactile devices like iPhones or iPads, certain features like text-to-speech or UI adjustment settings allow creatives with visual disabilities to remain involved. Unfortunately, I am not confident in my ability to implement accessibility features that would ensure equal opportunity, since the majority of my focus will likely be on basic implementation and attempts toward designing an intuitive and simple interface. Ultimately, I anticipate the final design of the software to be exclusive and geared toward society's normative view of the typical abled user. 

\section{A Lack of Interpersonal Interactions}
Creely and Henriksen's insistence on the importance of interpersonal interactions requires an understanding of the function of technology within our society, and its potential harmful effects on an individual's mental state. The invention of the smartphone, a technological device that fits in our back pocket and can be used as an intermediary or substitute for most necessary actions in society has resulted in a modern age that values efficiency over incubation and creates insulation that "stymies the kinds of difference and conversation that inspire creative thought or solutions" (Creely and Henriksen). In essence, as much as technology is advertised as a tool for connection, it is a brilliant facilitator of seclusion and alienation. 

Visualizing this project as a tool for a single designer, without any function for collaboration, such as simultaneous creation or gallery-sharing, means that even accidentally, I will have created another tool that prioritizes the individual over collective engagement. I ultimately believe that this will be detrimental to the creative energies of my users. It is difficult to make a generalizing claim, but primary sources indicate the outside social forces can have a huge impact on the resulting creative output (Amabile et al.). Focusing on the connection between creativity and social interaction, research asserts that the stereotypes associating creativity with "solitary, spontaneous, casual and sudden processes" are misguided. Instead, creativity is typically found to "develop in interaction with others"(Elisondo).

Of course, users would be free to share their creations on social media sites to receive engagement and feedback. However, the implications of software without social features are a preference for a solitary process of creation. Without the ability to gain feedback and collaborate throughout the process of creation, I feel that the software would ultimately be more limiting than it would be freeing. Pressuring users to view their creations through an individualistic lens, necessitating a veneer of completion before feeling confident enough to share has the potential to result in creative stagnation. Returning to the issues of technological solutionism promoting a singular, algorithmic process of doing, the fault in the program would be a failure to recognize the relevance of social interaction. And ultimately, I would be unprepared to address this issue. 

\section{Upkeep and Maintenance} 
The final issue I wish to address is the possibility of upkeep and maintenance after the initial completion of a prototype. While I have initially envisioned my project as a stand-alone software without the need for updates, I have come to realize the resultant issues of this viewpoint. Firstly, a stagnation of the project's development would not be able to properly address the evolving needs of my users. In a certain sense, it would be irresponsible to promise users an application meant to address a problem without considering the likelihood of holes in my solution. If users come to find that the software is unable to suit their needs, and I am unable to dedicate more time to the project after its completion, there is a possibility that the project could be disadvantageous. If users grow accustomed to inadequate software to the point that it hurts their creative process and ultimate product, then the software end up doing more harm than good. A shallow promise to address my users' needs without following through would be misleading and unethical. 

However, lack of upkeep does not only pertain to an inability to create a functional program. It also extends to ensuring that the software remains reliable and can adapt to changing technology. In the future, as machines and computers continue to progress, the inevitability that my project will be unable to perform on new platforms or devices could result in the loss of creative progress. For users in the middle of the design process or storing idea drafts of scenes, a shutdown of the software would be disastrous. The reality of the software is that it should encourage continuous usage, both of a singular project, but also allow users to generate new ideas for future projects. 

If I were to dedicate time to the maintenance of this project, beyond the scope of the senior comprehensive project, then I would want to ensure some sort of payment for my time. However, it is difficult to imagine an ethical way to monetize the project. Since the initial implementation would be free-to-use, and as a solo developer, I doubt I would be able to make much money trying to market the product. Therefore, the only two ways that I can foresee myself earning money are through data collection or creative royalties. 

Data collection is only ethical when individuals provide informed consent. An individual owns their personal information, and in collecting data it is necessary to be transparent, to protect users' privacy, and to prevent disparate impact or "inadvertent harm to individuals or groups of people" (Cote). In sum, ethically dealing with individual data requires a certain degree of cybersecurity and an assurance that sensitive information will be treated appropriately. Unfortunately, I am not familiar with cybersecurity and it would be difficult for me to ensure that data buyers would be responsible. I would be unable to prevent outcomes like identity theft. Therefore, I believe that it would be unethical to pursue this method given my limitations. 

The possibility of creative royalties is a complex issue to enforce. The game engine Unreal utilizes a notification of release form and requires developers to track their gross revenue to calculate royalties(“Unreal Engine | Commercial Game Deployment Guidelines”). However, this is essentially only applicable to developers releasing commercially successful games built using the game engine as the primary foundation. A set design visualizer, asking for creative royalties as the foundation of creative vision seems an inadequate analogy as it disregards both the designer's originality and the world involved in physically constructing the set. What's more, my targeted audience is students and amateur designers, who likely aren't making much if any profit. Even if an expert designer were to use my software, it would be difficult for me to ensure that users are honestly reporting, as I would have to be able to track performances worldwide. It is difficult to tell whether the quality of my project will be high enough to merit financial payment.
\section{Conclusion}
In conclusion, while considering the variety of ethical factors that concern my project, it is difficult to be sure that my original purpose of benefitting creators could be properly pursued. Regarding the high potential of ethical abuses committed by users, I feel hesitant to create a project that would facilitate harm or perpetrate offensive or threatening content. I also want to avoid the trap of technological solutionism and feel that the limited scope of my project would result in enforcing a creative process that limits the physical involvement of users and isolates their creative energy. Finally, my inability to see the project through beyond the initial proposal leads me to believe that it would be irresponsible to create a set design software with misguided intentions. Taking all these facets into account, it is clear that it is beyond my current capabilities to create an ethical version of this project.  
\section{References}
\begin{hangparas}{.25in}{1}
Amabile, Teresa M., et al. Creativity in Context. Routledge, 4 May 2018.

Cote, Catherine. “5 Principles of Data Ethics for Business.” Harvard Business School Online, 16 Mar. 2021, online.hbs.edu/blog/post/data-ethics.

Creely, Edwin, and Danah Henriksen. Creativity and Digital Technologies. 1 Sept. 2019, pp. 1–6, https://doi.org/10.1007/978-981-13-2262-4143-1.

Dow Schull, Natasha. “The Folly of Technological Solutionism: An Interview with Evgeny Morozov.” Public Books, 9 Sept. 2013, www.publicbooks.org/the-folly-of-technological-solutionism-an-interview-with-evgeny-morozov/.

“ELECTRONIC ARTS USER AGREEMENT.” Ea.com, 2022, tos.ea.com/legalapp/webterms/US/en/PC. Accessed 18 Apr. 2024.

Elisondo, Romina. “Creativity Is Always a Social Process.” Creativity. Theories – Research - Applications, vol. 3, no. 2, 1 Dec. 2016, pp. 194–210, https://doi.org/10.1515/ctra-2016-0013.

“Legal.” Www.adobe.com, www.adobe.com/legal/terms.html. Accessed 18 Apr. 2024.

“Unreal Engine | Commercial Game Deployment Guidelines.” Unreal Engine, www.unrealengine.com/en-US/release.
\end{hangparas}
\end{document}